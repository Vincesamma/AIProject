\documentclass[11pt]{article}
\usepackage[utf8]{inputenc}
\usepackage{amsmath, amssymb}
\usepackage{graphicx}
\usepackage{hyperref}
\usepackage{geometry}
\geometry{margin=1in}
\usepackage{authblk}

\title{Allocazione Adattiva ed Equa di Risorse di Protezione Sociale in Contesti di Rischio Climatico: Un Approccio basato su CSP e Intelligenza Artificiale}
\author[1]{Your Name}
\author[2]{Coauthor Name}
\affil[1]{Your Institution, Department, Email}
\affil[2]{Coauthor Institution, Department, Email}
\date{\today}

\begin{document}

\maketitle

\begin{abstract}
    Questo studio esplora il potenziale dell'intelligenza artificiale per affrontare il problema dell'allocazione di risorse di protezione sociale in contesti caratterizzati da rischio climatico dinamico. 
    Integrando vincoli operativi, esigenze di equità, ed efficienza, proponiamo un modello basato su Soft CSP per ottimizzare decisioni complesse, adattandosi ai dati ambientali e socioeconomici in tempo reale.
\end{abstract}

\section{Introduction}

Diversi documenti esaminano l'intersezione cruciale tra la protezione sociale (SP) e l'adattamento al cambiamento climatico (CCA), specialmente nel contesto della gestione delle vulnerabilità socioeconomiche e della sicurezza alimentare in paesi in via di sviluppo come il Brasile. 
Le sfide poste dal cambiamento climatico includono l'aumento della frequenza e intensità di eventi meteorologici estremi, disastri e la degradazione degli ecosistemi, che a loro volta portano a povertà, insicurezza alimentare, perdita di beni e mezzi di sussistenza.

\section{Background e Motivazione}
I sistemi di protezione sociale sono identificati come uno strumento politico chiave per gestire i rischi derivanti dal cambiamento climatico, complementando le misure di risposta ai disastri, adattamento e mitigazione. 
Tuttavia, l'integrazione strategica tra politiche SP e CCA risulta spesso frammentaria o assente.  quindi cruciale sviluppare approcci strutturati e basati sui dati per ottimizzare tali sistemi.

\section{Problem Formulation}
\textbf{Problema:} Come ottimizzare l'allocazione limitata di risorse di protezione sociale tra popolazioni vulnerabili geograficamente disperse, in modo che l'allocazione sia efficiente ed equa, tenendo conto dei rischi climatici dinamici e dei vincoli esistenti?

Questa problematica è particolarmente adatta a essere modellata con tecniche di Constraint Satisfaction Problem (CSP) e Soft CSP, che consentono di formalizzare decisioni complesse soggette a vincoli e preferenze multiple.

\section{Approccio Basato su CSP/Soft CSP}
Un CSP è definito da:
\begin{itemize}
\item Variabili: decisioni di allocazione (es. risorse per regione e periodo).
\item Domini: valori possibili (es. quantità, percentuali).
\item Vincoli Hard: budget, capacità operative, idoneità.
\item Vincoli Soft: equità, efficienza, rischio climatico.
\end{itemize}

Nel nostro contesto, proponiamo un \textit{Adaptive Soft CSP Framework} dove i vincoli e i dati cambiano nel tempo, e le soluzioni vengono aggiornate periodicamente in base a nuove informazioni.

\section{Tecniche di Intelligenza Artificiale}
\begin{itemize}
\item \textbf{Problem Formulation AI:} modellazione di stati, azioni, transizioni.
\item \textbf{CSP e Soft CSP:} struttura naturale per il problema.
\item \textbf{Algoritmi di Ottimizzazione:} metaeuristiche (es. Simulated Annealing, Genetic Algorithms) per esplorare lo spazio delle soluzioni.
\item \textbf{Agente Intelligente:} un sistema che percepisce, decide e agisce, aggiornando dinamicamente la soluzione.
\end{itemize}

\section{Benefici del Modello Proposto}
\begin{itemize}
\item Decisioni evidence-based
\item Gestione della complessità
\item Equità modellata esplicitamente
\item Adattabilità a contesti dinamici
\item Trasparenza nelle decisioni
\item Simulazione di scenari politici e climatici
\end{itemize}

\section{Conclusioni}
L'integrazione tra protezione sociale e adattamento climatico è essenziale per aumentare la resilienza delle popolazioni vulnerabili. 
Un approccio AI basato su CSP/Soft CSP consente di sviluppare strumenti decisionali dinamici, equi ed efficienti.

\bibliographystyle{plain}
\bibliography{references}  % expects a references.bib file

\end{document}
