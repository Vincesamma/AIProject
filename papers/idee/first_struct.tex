\documentclass[a4paper,12pt]{article}
\usepackage[utf8]{inputenc}
\usepackage[italian]{babel}
\usepackage{amsmath,amssymb}
\usepackage{graphicx}
\usepackage{hyperref}
\usepackage{geometry}
\geometry{margin=2.5cm}
\usepackage{enumitem}

\title{\textbf{Allocazione Intelligente delle Risorse di Protezione Sociale in Contesti di Rischio Climatico Dinamico}}
\author{Progetto basato su tecniche di Intelligenza Artificiale}
\date{\today}

\begin{document}

\maketitle

\section*{Introduzione}
Basandosi sulle informazioni contenute nel documento \textit{"Paper.pdf"} e integrando concetti da altre fonti, questo progetto propone un metodo basato sull'Intelligenza Artificiale per affrontare la problematica dell'allocazione di risorse di protezione sociale in contesti colpiti da rischi climatici dinamici.

\section*{Descrizione del Problema}
Il problema fondamentale riguarda l'ottimizzazione dell'allocazione limitata di risorse di protezione sociale tra popolazioni vulnerabili disperse geograficamente. L'obiettivo è ottenere un'allocazione efficiente ed equa, tenendo conto di:

\begin{itemize}
    \item Rischi climatici dinamici (es. siccità, inondazioni).
    \item Vincoli operativi e di bilancio.
    \item Esigenze di equità e adattabilità.
\end{itemize}

Le sfide climatiche aggravano povertà e insicurezza alimentare, rendendo essenziale l'integrazione tra protezione sociale (SP) e adattamento al cambiamento climatico (CCA), spesso però frammentata.

\section*{Modellazione del Problema}
Il problema può essere modellato come un \textbf{Adaptive Soft Constraint Satisfaction Problem (Soft CSP)}, che permette di includere:

\begin{itemize}
    \item \textbf{Vincoli Hard:} budget, capacità logistiche, criteri di ammissibilità.
    \item \textbf{Vincoli Soft:} equità, efficienza, priorità basate sul rischio climatico.
\end{itemize}

Questa formulazione consente una rappresentazione flessibile del problema e la possibilità di aggiornare dinamicamente i vincoli e i dati.

\section*{Tecniche di Intelligenza Artificiale Utilizzate}

\subsection*{1. Problem Formulation AI}
Per modellare stati, azioni e transizioni nel processo decisionale.

\subsection*{2. CSP e Soft CSP}
Struttura naturale per formalizzare decisioni complesse con vincoli multipli. Soft CSP permette di modellare preferenze e priorità flessibili.

\subsection*{3. Algoritmi di Ottimizzazione}
Utilizzo di \textbf{metaeuristiche} come:
\begin{itemize}
    \item \textbf{Algoritmi Genetici (GA):} generazione e selezione evolutiva di soluzioni.
    \item \textbf{Simulated Annealing (SA):} utile per evitare ottimi locali.
\end{itemize}

\subsection*{4. Agente Intelligente}
Sistema autonomo che percepisce l’ambiente, aggiorna il modello, risolve il problema e agisce:

\begin{itemize}
    \item \textbf{Percezione:} acquisizione dati da fonti climatiche, socioeconomiche e sul campo.
    \item \textbf{Aggiornamento:} adattamento dinamico del Soft CSP in base ai cambiamenti.
    \item \textbf{Risoluzione:} ricerca di soluzioni ottimali o sub-ottimali.
    \item \textbf{Azione:} emissione di decisioni di allocazione.
\end{itemize}

\section*{Progetto Applicativo Proposto}

\textbf{Scenario:} Area geografica soggetta a rischi climatici ricorrenti (es. il Brasile semi-arido).

\begin{itemize}
    \item \textbf{Dominio:} Allocazione di risorse come aiuti alimentari, finanziari o rifugi temporanei.
    \item \textbf{Modello:} Adaptive Soft CSP con:
        \begin{itemize}
            \item \textbf{Variabili:} decisioni di allocazione per comunità o famiglie.
            \item \textbf{Domini:} valori interi, binari, continui.
            \item \textbf{Vincoli Hard:} budget, trasporti, capacità logistiche.
            \item \textbf{Vincoli Soft:} equità, efficienza, priorità basate su rischio e danno.
        \end{itemize}
    \item \textbf{Agente Intelligente:}
        \begin{enumerate}[label=\alph*)]
            \item \textbf{Percezione} di dati climatici, sociali ed economici.
            \item \textbf{Aggiornamento del Modello} con vincoli e priorità modificabili.
            \item \textbf{Risoluzione del CSP} con GA o SA.
            \item \textbf{Azione} tramite raccomandazioni operative.
        \end{enumerate}
    \item \textbf{Equità:} metriche incorporate nei vincoli Soft e nelle funzioni obiettivo.
    \item \textbf{Adattabilità:} aggiornamento continuo del modello per risposte in tempo reale.
\end{itemize}

\section*{Conclusioni}
L’integrazione di tecniche di IA consente lo sviluppo di strumenti decisionali \textit{evidence-based}, adattivi, equi ed efficienti, cruciali per la gestione di risorse di protezione sociale in scenari climatici dinamici. Il sistema proposto offre:

\begin{itemize}
    \item Maggiore efficacia nella risposta.
    \item Migliore trasparenza e tracciabilità.
    \item Capacità di simulare scenari e supportare decisioni strategiche.
\end{itemize}

\end{document}
