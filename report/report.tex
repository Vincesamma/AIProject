\documentclass[letterpaper]{article}
\usepackage[submission]{aaai25}
\usepackage{mathptmx}
\usepackage{helvet}
\usepackage{courier}
\usepackage[hyphens]{url}
\usepackage{graphicx}
\urlstyle{rm}
\def\UrlFont{\rm}
\usepackage{natbib}
\usepackage{caption}
\usepackage{algorithm}
\usepackage{algorithmic}

\frenchspacing
\setlength{\pdfpagewidth}{8.5in}
\setlength{\pdfpageheight}{11in}

\title{Allocazione Adattiva ed Equa di Risorse di Protezione Sociale in Contesti di Rischio Climatico Dinamico: Un Approccio basato su CSP e Intelligenza Artificiale}
\author{}
\date{}
\affiliations{}

\begin{document}

\maketitle

\begin{abstract}
    Questo studio esplora il potenziale dell'intelligenza artificiale (AI) per affrontare la problematica cruciale dell'allocazione di risorse di protezione sociale (SP) in contesti caratterizzati da rischio climatico dinamico. Integrando vincoli operativi, esigenze di equità, ed efficienza, proponiamo un modello basato su Soft Constraint Satisfaction Problem (Soft CSP) per ottimizzare decisioni complesse. L'approccio si adatta ai dati ambientali e socioeconomici in tempo reale e si avvale di algoritmi di ottimizzazione metaeuristica e di un agente intelligente per implementare decisioni basate sull'evidenza. Questo framework mira a fornire uno strumento decisionale robusto per aumentare la resilienza delle popolazioni vulnerabili di fronte alle crescenti minacce climatiche.
\end{abstract}

\section{Introduzione}
L'intersezione tra protezione sociale e adattamento al cambiamento climatico (CCA) è una tematica cruciale, specialmente nel contesto della gestione delle vulnerabilità socioeconomiche e della sicurezza alimentare nei paesi in via di sviluppo. Il cambiamento climatico (CC) aggrava le sfide esistenti attraverso l'aumento della frequenza e dell'intensità di eventi meteorologici estremi, disastri naturali e la degradazione degli ecosistemi. Questi fenomeni si traducono in povertà, insicurezza alimentare, e perdita di beni e mezzi di sussistenza per le popolazioni già vulnerabili.

I disastri legati alle inondazioni, ad esempio, oltre a causare vittime, danno origine a malattie trasmesse dall'acqua e dai vettori, rendendo la rapida distribuzione di materiali di soccorso e le operazioni di ricollocamento fondamentali. Analogamente, l'allocazione di risorse di emergenza come rifugi temporanei è una questione chiave per chi risponde alle emergenze, ma è spesso complicata da considerazioni inadeguate sui bisogni socioeconomici, sulla vulnerabilità delle sistemazioni temporanei a potenziali ulteriori impatti del disastro e sull'inefficienza nel controllare i costi pubblici.

In questo scenario complesso e dinamico, i sistemi di protezione sociale emergono come strumenti politici essenziali. Tuttavia, l'integrazione strategica tra politiche SP e CCA è spesso frammentaria o assente, limitando l'efficacia complessiva delle risposte. È quindi indispensabile sviluppare approcci strutturati, basati sui dati e adattivi per ottimizzare l'allocazione delle risorse limitate in modo efficiente ed equo. Il nostro progetto propone l'utilizzo dell'Intelligenza Artificiale per affrontare direttamente questa sfida.

\section{Background e Motivazione}
I sistemi di protezione sociale sono ampiamente riconosciuti come uno strumento politico chiave per gestire i rischi derivanti dal cambiamento climatico. Essi possono integrare e complementare le misure di risposta ai disastri, adattamento e mitigazione. Ad esempio, i trasferimenti di denaro regolari possono contribuire a ridurre le pressioni sugli ecosistemi, aiutando le famiglie povere e vulnerabili a diminuire lo sfruttamento eccessivo delle risorse naturali per la sussistenza. Possono anche essere utilizzati per mitigare l'impatto negativo della rimozione dei sussidi ai combustibili fossili, come nel caso del ``Bono Gas Hogar'' nella Repubblica Dominicana. Tuttavia, i trasferimenti di denaro possono anche avere impatti ambientali complessi, non sempre positivi, come un aumento del consumo di beni che richiedono un uso intensivo del suolo in contesti con infrastrutture limitate e accesso al mercato. Questo sottolinea l'importanza di comprendere come le popolazioni vivono i rischi climatici per informare la progettazione della protezione sociale e degli interventi complementari, evitando il rischio di impatti negativi non intenzionali.

Nonostante il potenziale, l'integrazione concettuale e pratica tra SP e CCA è limitata. Una revisione della documentazione ufficiale in Brasile, ad esempio, ha mostrato una limitata interazione tra le questioni di SP e sicurezza alimentare (FS) nei documenti che affrontano il CC. La mancanza di menzione di FS o programmi SP (come PAA o Bolsa Família) nel principale documento politico sul CC in Brasile (PNMC) è stata interpretata come un segno di comunicazione limitata o mancanza di consapevolezza sull'importanza di considerare congiuntamente queste sfere. Le distinte capacità umane e i background nei diversi ministeri (SP, ambiente, scienza e tecnologia) contribuiscono a prospettive diverse sulla priorità relativa di determinate questioni.

In questo contesto, la necessità di sviluppare approcci strutturati e basati sui dati per ottimizzare i sistemi SP è chiara. L'allocazione di risorse sanitarie, ad esempio, spesso avviene sulla base di forze politiche o della continuazione dello status quo, portando a sprechi e perdita di potenziale salute della popolazione. I sistemi sanitari, per servire al meglio le persone, devono operare in modo efficiente ed equo. Similmente, nel contesto dei disastri naturali, la distribuzione dei soccorsi di emergenza non è stata sempre adeguatamente affrontata in studi precedenti. L'efficacia e l'equità del sistema di distribuzione complessivo sono fondamentali per evitare di trascurare aree critiche ma difficili da raggiungere. Modelli basati sull'ottimizzazione matematica, come quelli utilizzati per l'allocazione di risorse sanitarie o la distribuzione di soccorsi post-disastro, possono fornire raccomandazioni basate sull'evidenza, considerando obiettivi politici, interazioni tra interventi, vincoli del mondo reale e budget.

L'obiettivo è passare da decisioni reattive e potenzialmente inefficienti o inique a un sistema proattivo e adattivo che utilizzi le informazioni disponibili (ambientali, sociali, economiche, logistiche) per guidare le decisioni di allocazione delle risorse SP.

\section{Formulazione del Problema}
Il problema fondamentale che il nostro progetto mira ad affrontare è: \textbf{Come ottimizzare l'allocazione limitata di risorse di protezione sociale tra popolazioni vulnerabili geograficamente disperse, in modo che l'allocazione sia efficiente ed equa, tenendo conto dei rischi climatici dinamici e dei vincoli esistenti?}.

Questo problema si distingue per diverse caratteristiche chiave che lo rendono complesso:
\begin{itemize}
    \item \textbf{Scarsità delle risorse:} Le risorse di protezione sociale (come aiuti alimentari, trasferimenti finanziari, rifugi temporanei, assistenza sanitaria) sono limitate.
    \item \textbf{Dispersione geografica:} Le popolazioni vulnerabili sono spesso distribuite su ampie aree, rendendo la logistica e il trasporto elementi critici. Nei contesti di soccorso in caso di inondazione, la rapidità di distribuzione e ricollocamento è essenziale. La distanza tra i blocchi e i rifugi, o tra i depositi centrali e i rifugi, è un fattore determinante nei modelli di ottimizzazione logistica.
    \item \textbf{Rischi climatici dinamici:} L'intensità e la localizzazione dei rischi (siccità, inondazioni, ecc.) cambiano nel tempo. Un approccio deve essere in grado di adattarsi a queste variazioni.
    \item \textbf{Molteplicità di obiettivi:} Non si cerca solo l'efficienza (minimizzare i costi o il tempo), ma anche l'equità nell'allocazione. Modelli multi-obiettivo nella distribuzione dei soccorsi considerano tipicamente obiettivi economici (costo minimo), di efficacia (tempo minimo di viaggio) e di equità (massima soddisfazione).
    \item \textbf{Vincoli complessi:} Esistono diverse limitazioni pratiche, tra cui budget disponibili, capacità dei veicoli e dei rifugi, e criteri di ammissibilità per le risorse.
\end{itemize}

Affrontare efficacemente questo problema richiede un framework che possa rappresentare adeguatamente tutti questi elementi e trovare soluzioni soddisfacenti anche in condizioni di incertezza e cambiamento.

\section{Modellazione del Problema}
Per rappresentare formalmente questo problema, proponiamo l'utilizzo di un \textbf{Adaptive Soft Constraint Satisfaction Problem (Soft CSP)}. Questa formulazione è particolarmente adatta perché permette di gestire non solo vincoli rigidi (che devono essere soddisfatti in ogni caso) ma anche preferenze o vincoli flessibili (che si cerca di soddisfare al meglio). Inoltre, il framework è ``adattivo'' in quanto consente ai vincoli e ai dati di cambiare nel tempo, con le soluzioni che vengono aggiornate periodicamente in base alle nuove informazioni percepite.

Nel contesto dell'allocazione delle risorse di protezione sociale:
\begin{itemize}
    \item \textbf{Variabili:} Le variabili del problema rappresentano le decisioni di allocazione delle risorse. Ad esempio, potrebbero indicare quali comunità o famiglie ricevono quali tipi e quantità di risorse, o quali rifugi temporanei vengono aperti e con quale capacità. Queste variabili possono assumere valori interi (es. numero di kit di soccorso), binari (es. un rifugio è aperto/chiuso), o continui (es. quantità di aiuto finanziario).
    \item \textbf{Domini:} I domini definiscono l'insieme dei possibili valori che ciascuna variabile può assumere.
    \item \textbf{Vincoli Hard:} Questi sono i vincoli che devono essere rispettati per avere una soluzione valida. Includono:
          \begin{itemize}
              \item \textbf{Vincoli di Budget:} L'allocazione totale non può superare il budget disponibile per ciascun tipo di risorsa. Esempi di costi in operazioni di soccorso includono costi di apertura rifugi, costi di trasporto (per veicolo e per km), e costi di penalità per mancata evacuazione o mancata fornitura di soccorsi.
              \item \textbf{Vincoli Logistici e di Capacità:} Le risorse allocate a una località non possono superare la sua capacità di ricezione (es. capacità dei rifugi). La quantità totale di risorse distribuite non può superare la disponibilità nei depositi. La capacità dei veicoli utilizzati per il trasporto (in peso o volume) è un altro vincolo logistico.
              \item \textbf{Criteri di Ammissibilità:} Solo le popolazioni che soddisfano determinati criteri socioeconomici o di vulnerabilità possono ricevere determinate risorse.
          \end{itemize}
    \item \textbf{Vincoli Soft:} Questi vincoli riflettono gli obiettivi preferenziali del sistema e contribuiscono a definire l'ottimalità della soluzione. Includono:
          \begin{itemize}
              \item \textbf{Equità nell'Allocazione e negli Outcomes:} Garantire una distribuzione giusta delle risorse, tenendo conto delle diverse vulnerabilità e bisogni. Le metriche di equità possono essere incorporate esplicitamente. Studi sulla fairness nell'allocazione delle risorse critiche evidenziano l'importanza di considerare l'equità sia nell'allocazione (chi riceve cosa) sia negli outcomes (l'impatto della risorsa). Notabilmente, la fairness nell'allocazione e la fairness negli outcomes sono spesso incompatibili.
              \item \textbf{Efficienza:} Minimizzare i costi operativi (trasporto, gestione) e i tempi di consegna delle risorse. L'efficienza può essere misurata in termini di costo totale minimo o tempo di viaggio totale minimo.
              \item \textbf{Priorità basate sul Rischio e sul Danno:} Dare priorità alle aree o alle popolazioni più colpite o a maggior rischio in base agli eventi climatici in corso. L'identificazione delle aree prioritarie può basarsi su parametri socioeconomici, demografici e ambientali.
              \item \textbf{Massimizzazione della Soddisfazione:} Nei modelli multi-obiettivo per la distribuzione dei soccorsi, un obiettivo è massimizzare il livello minimo di soddisfazione tra i punti di domanda per ciascun articolo.
          \end{itemize}
\end{itemize}

La formulazione Soft CSP permette di bilanciare questi vincoli e obiettivi, trovando soluzioni che siano valide (rispettando i vincoli Hard) e che ottimizzino il grado di soddisfazione dei vincoli Soft (massimizzando efficienza ed equità, tenendo conto delle priorità).

\section{Tecniche di Intelligenza Artificiale Utilizzate}
Per risolvere l'Adaptive Soft CSP e implementare il sistema decisionale, proponiamo l'integrazione di diverse tecniche di Intelligenza Artificiale:

\begin{enumerate}
    \item \textbf{Problem Formulation AI:} Questa fase iniziale consiste nel modellare formalmente il problema in termini di stati, azioni e transizioni. Ad esempio, uno stato potrebbe rappresentare una configurazione attuale delle risorse allocate e delle condizioni climatiche; un'azione potrebbe essere una specifica decisione di allocazione (es. inviare N unità di aiuto alimentare alla comunità X); una transizione descrive come un'azione modifica lo stato attuale. Questo processo è fondamentale per strutturare il problema per le tecniche di risoluzione successive.

    \item \textbf{CSP e Soft CSP:} Come discusso, il Soft CSP è la struttura naturale scelta per formalizzare le decisioni complesse con vincoli multipli e priorità flessibili. La sua capacità di gestire la flessibilità la rende particolarmente adatta a problemi del mondo reale dove non tutti i vincoli sono rigidi.

    \item \textbf{Algoritmi di Ottimizzazione (Metaeuristiche):} Trovare la soluzione ottimale in un Soft CSP, specialmente di grandi dimensioni e con dati che cambiano dinamicamente, è un problema computazionalmente difficile. Le metaeuristiche sono algoritmi di ricerca che esplorano lo spazio delle soluzioni in modo efficiente. Proponiamo l'uso di:
          \begin{itemize}
              \item \textbf{Algoritmi Genetici (GA):} Ispirati all'evoluzione naturale, i GA lavorano con una ``popolazione'' di soluzioni candidate, rappresentate come stringhe (spesso binarie). Le nuove generazioni vengono prodotte tramite selezione, crossover e mutazione. Ogni stato è associato a un valore (fitness function) che misura la sua qualità rispetto agli obiettivi. I GA sono stati applicati con successo a problemi di ottimizzazione nella distribuzione di soccorsi, come l'allocazione di rifugi e la distribuzione di materiali.
              \item \textbf{Simulated Annealing (SA):} Questa metaeuristica si ispira al processo di ricottura dei metalli. Permette di ``saltare'' fuori dagli ottimi locali accettando occasionalmente mosse che peggiorano la soluzione, con una frequenza che diminuisce gradualmente nel tempo (``temperatura''). È dimostrato che, se la ``temperatura'' diminuisce abbastanza lentamente, SA può trovare un ottimo globale.
          \end{itemize}
          L'uso di GA o SA o una combinazione di metaeuristiche consente di esplorare efficacemente lo spazio delle soluzioni del Soft CSP per trovare soluzioni ottimali o sub-ottimali, anche in problemi complessi dove i metodi esatti potrebbero essere troppo lenti o impraticabili. Altre metaeuristiche, come la ricerca locale a fascio (local beam search) o gli algoritmi Bees, sono state impiegate in problemi simili di ottimizzazione per la distribuzione di soccorsi o l'assegnazione generalizzata.

    \item \textbf{Agente Intelligente:} Per integrare le tecniche sopra menzionate e interagire con l'ambiente dinamico, il framework si basa su un Agente Intelligente. Un agente intelligente è un sistema che percepisce il suo ambiente, prende decisioni e agisce per raggiungere i suoi obiettivi. Nel nostro contesto, l'agente:
          \begin{itemize}
              \item \textbf{Percepisce:} Acquisisce dati da diverse fonti, tra cui dati climatici (previsioni, eventi in corso), dati socioeconomici (vulnerabilità delle popolazioni, bisogni attuali) e dati sul campo (condizioni logistiche, capacità dei rifugi). La raccolta dati in contesti di emergenza include informazioni socioeconomiche per le aree colpite e matrici di distanza. I dati sui parametri per l'identificazione delle aree prioritarie includono popolazione, area, reddito pro capite, valore delle proprietà e fattori come drenaggio, urbanizzazione e servizi civici. Le matrici di distanza possono essere formate utilizzando mappe scalabili o sistemi GIS.
              \item \textbf{Aggiorna:} Adatta dinamicamente il modello Soft CSP in base ai cambiamenti nelle condizioni percepite. Se una nuova allerta meteo o una valutazione dei danni post-evento indicano un aumento del rischio in una certa area, i vincoli Soft (es. priorità) vengono modificati. Se la disponibilità di risorse o la capacità logistica cambiano, i vincoli Hard vengono aggiornati.
              \item \textbf{Risolve:} Utilizza gli algoritmi di ottimizzazione (GA/SA) per trovare la soluzione ottimale o sub-ottimale per il Soft CSP aggiornato.
              \item \textbf{Agisce:} Emette le decisioni di allocazione delle risorse sotto forma di raccomandazioni operative. Queste decisioni guidano l'invio di aiuti, l'apertura di rifugi, l'assegnazione di trasferimenti di denaro, ecc.
          \end{itemize}
          L'architettura dell'agente include abilità, conoscenze pregresse, obiettivi, stimoli (correnti e passati), e uno stato di credenza utilizzato per scegliere le azioni. La trasduzione, che mappa i percetti ai comandi, può essere implementata da un controller. Un approccio gerarchico ai controller può rendere il sistema più veloce. L'agente mantiene uno stato di credenza che codifica la storia dei percetti.
\end{enumerate}

\section{Progetto Applicativo Proposto: Scenario nel Semi-arido Brasiliano}
Per illustrare l'applicazione di questo framework, proponiamo uno scenario basato sull'area geografica del semi-arido brasiliano. Questa regione è particolarmente pertinente in quanto è una delle aree più vulnerabili del paese agli shock socioeconomici e climatici, in particolare alla siccità (``seca/estiagem''), ed è dove i programmi di protezione sociale basati sulla sicurezza alimentare hanno un impatto più rilevante.

\begin{itemize}
    \item \textbf{Dominio:} L'applicazione si concentra sull'allocazione di risorse di protezione sociale specifiche per questo contesto, come aiuti alimentari (kit, accesso a cibo attraverso programmi come PAA e PNAE), trasferimenti finanziari (come il Programma Bolsa Família, menzionato nelle referenze), e rifugi temporanei in caso di eventi estremi.

    \item \textbf{Modello:} Viene sviluppato un Adaptive Soft CSP per questo dominio.
          \begin{itemize}
              \item \textbf{Variabili:} Le decisioni potrebbero includere: quale percentuale della popolazione idonea in ciascun comune riceve un trasferimento di denaro, quante tonnellate di cibo vengono inviate a ciascun centro di distribuzione, o quali scuole fungono da rifugi temporanei e con quale capacità.
              \item \textbf{Domini:} I domini delle variabili dipenderanno dal tipo di risorsa (quantità, binario per apertura/chiusura rifugio, ecc.).
              \item \textbf{Vincoli Hard:} Budget totale disponibile per i programmi SP. Capacità di stoccaggio e distribuzione dei centri logistici. Capacità massima dei rifugi temporanei. Disponibilità di veicoli e limiti sulla quantità trasportabile. Criteri di ammissibilità per l'accesso ai programmi (es. livello di reddito, esposizione al rischio).
              \item \textbf{Vincoli Soft:} Equità nella distribuzione delle risorse tra le diverse comunità (es. minimizzare le differenze nel livello di supporto pro-capite, o garantire che le aree più povere ricevano un supporto proporzionale alla loro vulnerabilità). Efficienza logistica (es. minimizzare i costi di trasporto totale, o il tempo totale per raggiungere le popolazioni più vulnerabili). Priorità basate sul rischio attuale e previsto di siccità o inondazioni. Queste priorità possono basarsi sulla probabilità degli eventi climatici e sulla vulnerabilità socioeconomica delle diverse aree.
          \end{itemize}

    \item \textbf{Agente Intelligente:} Un agente monitora continuamente l'ambiente.
          \begin{itemize}
              \item \textbf{Percezione:} Raccoglie dati sulle previsioni climatiche (es. probabilità di siccità nei prossimi mesi, intensità delle piogge previste), dati socioeconomici (es. indici di povertà aggiornati a livello locale, prezzi del cibo sui mercati locali, dati sull'occupazione), e dati logistici in tempo reale (es. condizioni delle strade, disponibilità di veicoli, stato degli stock nei depositi). La percezione continua è fondamentale data la natura dinamica del rischio climatico.
              \item \textbf{Aggiornamento:} In base ai dati percepiti, l'agente aggiorna il Soft CSP. Ad esempio, un peggioramento delle previsioni di siccità in una micro-regione potrebbe aumentare la priorità (vincolo Soft) per l'invio di aiuti alimentari in quell'area. Un aumento inaspettato dei costi del carburante potrebbe influenzare i vincoli Hard e Soft relativi ai costi di trasporto.
              \item \textbf{Risoluzione:} L'agente esegue gli algoritmi di ottimizzazione (GA o SA) per trovare la migliore allocazione di risorse dato il Soft CSP aggiornato.
              \item \textbf{Azione:} Genera raccomandazioni operative specifiche, come ``Allocare X unità di risorsa Y alla comunità Z, utilizzando il percorso logistico W''.
          \end{itemize}

    \item \textbf{Equità e Adattabilità:} L'equità è esplicitamente modellata nei vincoli Soft e nelle funzioni obiettivo del CSP. L'adattabilità è garantita dall'aggiornamento continuo del modello in base ai dati percepiti e dalla risoluzione periodica del CSP, consentendo risposte rapide e informate ai cambiamenti nel rischio e nei bisogni. La capacità di adattare i programmi SP alla variabilità climatica e ai cicli produttivi è considerata un progresso importante.
\end{itemize}

Questo scenario applicativo nel semi-arido brasiliano fornisce un contesto concreto per testare e validare l'efficacia del framework proposto nel gestire le complessità dell'allocazione adattiva ed equa delle risorse SP in risposta ai rischi climatici dinamici.

\section{Conclusioni}
L'integrazione tra protezione sociale e adattamento climatico è riconosciuta come essenziale per aumentare la resilienza delle popolazioni vulnerabili di fronte alle crescenti minacce poste dal cambiamento climatico. Tuttavia, l'attuale frammentazione tra politiche SP e CCA limita l'efficacia delle risposte.

Il nostro progetto propone un approccio basato sull'Intelligenza Artificiale che utilizza un framework Adaptive Soft CSP, algoritmi di ottimizzazione metaeuristica (come GA e SA) e un Agente Intelligente per affrontare questa sfida. Questo approccio consente di sviluppare strumenti decisionali dinamici, equi ed efficienti, superando le limitazioni dei metodi di allocazione basati sullo status quo o unicamente su considerazioni politiche.

I benefici del modello proposto includono:
\begin{itemize}
    \item \textbf{Decisioni basate sull'evidenza:} L'allocazione si basa sull'analisi rigorosa dei dati e dei vincoli.
    \item \textbf{Gestione della complessità:} Il framework è progettato per gestire l'elevato numero di variabili e vincoli interdipendenti tipici dei problemi di allocazione su larga scala.
    \item \textbf{Equità modellata esplicitamente:} Le preoccupazioni relative alla giustizia distributiva sono integrate formalmente nel processo decisionale attraverso i vincoli Soft.
    \item \textbf{Adattabilità a contesti dinamici:} La capacità di aggiornare il modello in tempo reale in risposta ai cambiamenti ambientali e socioeconomici è cruciale in scenari climatici volatili.
    \item \textbf{Trasparenza nelle decisioni:} Il processo decisionale basato sul modello può offrire maggiore chiarezza e tracciabilità rispetto a metodi meno strutturati.
    \item \textbf{Simulazione di scenari:} Il modello può essere utilizzato per simulare l'impatto di diverse politiche di allocazione o scenari climatici futuri, supportando la pianificazione strategica.
\end{itemize}

L'implementazione di un tale sistema richiederebbe lo sviluppo di solide infrastrutture per la raccolta e l'integrazione dei dati (climatici, socioeconomici, logistici). Le esperienze nella modellazione della distribuzione di soccorsi suggeriscono la necessità di basi dati complete e sistemi di informazione geografica. Ulteriori studi dovrebbero concentrarsi sui passi dettagliati per l'implementazione pratica del modello, sulla gestione dell'incertezza nei parametri di input (potenzialmente utilizzando approcci robusti o fuzzy) e sull'integrazione completa con le fasi di pianificazione pre-disastro. Nonostante le sfide, l'approccio AI proposto offre un percorso promettente per migliorare significativamente l'efficacia e l'equità della protezione sociale nel salvaguardare le popolazioni più vulnerabili di fronte alle minacce del cambiamento climatico.

\end{document}