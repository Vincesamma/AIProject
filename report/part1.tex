\begin{abstract}
    This study explores the potential of artificial intelligence (AI) to
    address the critical issue of allocating social protection (SP)
    resources in contexts characterized by dynamic climate risk. By
    integrating operational constraints, equity requirements, and
    efficiency considerations, we propose a model based on the Soft
    Constraint Satisfaction Problem (Soft CSP) to optimize complex
    decision-making. The approach adapts to real-time environmental
    and socioeconomic data and leverages metaheuristic optimization
    algorithms and an intelligent agent to implement evidence-based
    decisions. This framework aims to provide a robust
    decision-making tool to enhance the resilience of vulnerable
    populations in the face of increasing climate threats.
\end{abstract}

\section{Introduction}
The intersection between social protection and \textbf{climate
    change adaptation (CCA)} is a crucial topic, especially in the
context of managing socioeconomic vulnerabilities and food security
in developing countries. \textbf{Climate change (CC)} exacerbates
existing challenges by increasing the frequency and intensity of
extreme weather events, natural disasters, and ecosystem degradation.
These phenomena lead to poverty, food insecurity, and the loss of
assets and livelihoods for already vulnerable populations.

Floods, for example, in addition to causing casualties, can
facilitate the spread of diseases transmitted through contaminated
water and insects, making the rapid distribution of aid and the
relocation of affected populations essential. Similarly, the
allocation of emergency resources such as temporary shelters is a
central issue for emergency management professionals. However, this
process is often complicated by a lack of attention to the
socioeconomic needs of affected people, the vulnerability of
temporary accommodations to potential new disaster impacts, and
inefficiencies in public cost management.

In this complex and dynamic scenario, social protection systems
emerge as essential policy tools. However, the strategic integration
between SP and CCA policies is often fragmented or absent, limiting
the overall effectiveness of responses. It is therefore essential
to develop structured, data-driven, and adaptive approaches to
optimize the allocation of limited resources in an efficient and
equitable way. Our project proposes the use of Artificial
Intelligence to directly address this challenge.

\section{Background and Motivation}
Social protection systems are widely recognized as a key policy tool
for managing risks associated with climate change. They can
complement and reinforce disaster response, adaptation, and
mitigation measures. For example, regular cash transfers can help
reduce pressure on ecosystems by enabling poor and vulnerable
households to lessen their overreliance on natural resources for
their livelihoods. They can also be used to mitigate the negative
impact of removing fossil fuel subsidies, as in the case of the
"Bono Gas Hogar" in the Dominican Republic. However, cash transfers
can also have complex environmental effects, not always positive,
such as increased consumption of land-intensive goods in contexts
with limited infrastructure and market access. This highlights the
importance of understanding how populations experience climate
risks in order to inform the design of social protection and
complementary interventions, avoiding the risk of unintended
negative impacts.

Despite their potential, the conceptual and practical integration
between SP and CCA remains limited. A review of official
documentation in Brazil, for example, showed limited interaction
between SP and food security (FS) issues in documents addressing
climate change. The absence of references to FS or SP programs
(such as PAA or Bolsa Família) in Brazil's main climate policy
document (PNMC) has been interpreted as a sign of limited
communication or lack of awareness about the importance of
considering these areas jointly. The distinct human capacities and
backgrounds within different ministries (SP, environment, science
and technology) contribute to differing perspectives on the
relative priority of specific issues.

In this context, the need to develop structured and data-driven
approaches to optimize SP systems is clear. The allocation of health
resources, for instance, often relies on political decisions or the
continuation of the status quo, leading to waste and loss of
potential population health benefits. Health systems must operate
efficiently and equitably to best serve the population. Similarly,
in the context of natural disasters, the distribution of emergency
relief has not always been adequately addressed in previous studies.
The overall effectiveness and fairness of the distribution system
are crucial to avoid overlooking critical but hard-to-reach areas.
Optimization models, such as those used for allocating health
resources or distributing post-disaster relief, can provide
evidence-based recommendations, taking into account policy
objectives, intervention interactions, real-world constraints, and
budget limitations.

The goal is to move from reactive and potentially inefficient or
inequitable decisions to a proactive and adaptive system that uses
available information (environmental, social, economic, logistical)
to guide SP resource allocation decisions.

\section{Problem Formulation}
The fundamental problem our project aims to address is:
\textbf{how can the limited allocation of social protection
    resources be optimized across geographically dispersed vulnerable
    populations, in a way that is both efficient and equitable, while
    accounting for dynamic climate risks and existing constraints?}

This problem is characterized by several key features that make it
complex:
\begin{itemize}
    \item \textbf{Resource scarcity:} social protection resources
          (such as food aid, cash transfers, temporary shelters, healthcare
          services) are limited.
    \item \textbf{Geographic dispersion:} vulnerable populations are
          often spread across large areas, making logistics and
          transportation critical elements. In flood relief contexts, the
          speed of distribution and relocation is essential. The distance
          between population clusters and shelters, or between central
          warehouses and shelters, is a key factor in logistic optimization
          models.
    \item \textbf{Dynamic climate risks:} the intensity and location
          of risks (droughts, floods, etc.) vary over time. The solution
          approach must be able to adapt to these changes.
    \item \textbf{Multiple objectives:} the goal is not only
          efficiency (minimizing cost or time) but also equity in
          allocation. Multi-objective models in relief distribution
          typically consider economic goals (minimum cost), effectiveness
          (minimum travel time), and fairness (maximum satisfaction).
    \item \textbf{Complex constraints:} there are various practical
          limitations, including available budgets, vehicle and shelter
          capacities, and eligibility criteria for receiving resources.
\end{itemize}

Effectively addressing this problem requires a framework capable of
accurately representing all these elements and finding satisfactory
solutions even under conditions of uncertainty and change.